% CRITERIA (Half a page)
	% - Main findigs
	% - What results mean
	% - Limitations of our method
	% - Comparison of our results (Campare to probabilistic path planning)
	% - Need for further studies
	
% OUTLINE
% - Main findings
	% - Worse distance than wavefront *complete (could use more info?)
	% - Expect success in reducing jerk *complete
		% - Find a metric
		% - Find a gold standard
	% - 100% success of finding a path w/o collisions
	% - Slower than wavefront *complete
	% - num generation *complete
	
\section{Discussion}

%% Length
The length of the path generated by the GA approach was on average XX\% longer than the path generated by the wavefront method to which it was compared. This is due to the fact that the wavefront method is an exhaustive search method, and will reliably find the absolute shortest path between two points. The polynomial genetic algorithm method fits a polynomial to the two points. In doing so, we see that the polynomial inherently takes a longer path to the end point as it oftern bulges to local maxima and minima.

%% Speed
Our algorithm was found to be much quicker at finding a solution than the wavefront method, which was used as the gold standard for finding the shortest path between two points. The wavefront method took an average of 90 minutes to analyze the shortest path between two points on any of the three test spaces used in this study. The genetic algorithm took an average of 5.7 seconds per search. This demostrates. that the genetic algorithm approach to path searching is faster than an exhaustive search method.

%% Number of generations
The algorithm was found to convege to a solution on average within 21.8 generations. This represents a quick convergence to the solution, and signaled that there was an effective diversity present in the algorithm.

%% Collisions
If was expected to be able to develop an algorithm that would find a valid path \(which contains no collisions\) 100\% of the time. The results from the experiment show that the algorithm infact finds a collision-free path XX\% of the time. This success rate is considered to be a success. In a real-life application, if the algorithm produces such an invalid path, it will be necissary to start the algorithm again from initialization. 

%% Jerk
A gold standard for the path with the shortest jerk could not be identified. Therefore, the metric by which the jerk of a path will be assessed will be to the results of the genetic algorithm when it is not trying to minimize jerk. 600 trials were performed with jerk minimization enabled. Another 600 trials were performed with the jerk criterion removed. It was found that the algorithm reduced jerk on average by XX\%. Although this is not a large reduction in jerk, it does represent an improvement. The lower jerk path will always be chosen by a robot operator over a similar path with higher jerk.



