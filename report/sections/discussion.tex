% CRITERIA (Half a page)
	% - Main findigs
	% - What results mean
	% - Limitations of our method
	% - Comparison of our results (Campare to probabilistic path planning)
	% - Need for further studies
	
% OUTLINE
% - Main findings
	% - Worse distance than wavefront
	% - Expect success in reducing jerk
		% - Find a metric
		% - Find a gold standard
	% - 100% success of finding a path w/o collisions
	% - Slower than wavefront *complete
	% - num generation
	
\section{Discussion}

%% Length
The length of the path generated by the GA approach was on average XX\% longer than the path generated by the wavefront method to which it was compared. This is due to the fact that the wavefront method is an exhaustive search method, and will reliably find the absolute shortest path between two points. The polynomial genetic algorithm method fits a polynomial to the two points. In doing so, we see that the polynomial inherently takes a longer path to the end point as it oftern bulges to local maxima and minima.

%% Speed
Our algorithm was found to be much quicker at finding a solution than the wavefront method, which was used as the gold standard for finding the shortest path between two points. The wavefront method took an average of 90 minutes to analyze the shortest path between two points on any of the three test spaces used in this study. The genetic algorithm took an average of 5.7 seconds per search. This demostrates. that the genetic algorithm approach to path searching is faster than an exhaustive search method.

