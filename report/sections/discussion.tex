% CRITERIA (Half a page)
	% - Main findigs
	% - What results mean
	% - Limitations of our method
	% - Comparison of our results (Campare to probabilistic path planning)
	% - Need for further studies
% OUTLINE - Main findings
	% - Worse distance than wavefront *complete (could use more info?)
	% - Expect success in reducing jerk *complete
		% - Find a metric
		% - Find a gold standard
	% - 100% success of finding a path w/o collisions
	% - Slower than wavefront *complete
	% - num generation *complete
%
Evaluating the performance of the genetic algorithm (GA) was complicated by its ability to consider many factors when determining the best path. These factors included path distance, vehicle dynamics and most importantly, obstacle avoidance. In order to determine a profile of the GA's performance, 600 independent tests were completed within 15 different environment configurations.

The length of the path generated by the GA approach was on average 0.7\% longer than the path generated by the wavefront method to which it was compared. The GA was able to determine a shorter path than the wavefront algorithm in 474 of 600 independent tests (79\%), while also minimizing manipulator jerk (sometimes at the cost of a longer path). The wavefront is a deterministic algorithm which will find a very short path around obstacles, with no consideration for vehicle dynamics. The GA works differently in that it is a stochastic algorithm that is sampling the environment in an effort to find the shortest path. As such, it can get stuck in a local minimum and not ultimately return the path of best possible fitness. While processes internal to the GA attempt to avoid these situations, it is not a perfect algorithm.

%% Speed
The GA was found to be significantly faster at finding a solution than the wavefront algorithm. The wavefront algorithm took an average of 17 minutes and 50 seconds to analyze the shortest path between two points in any of the three test spaces used in this study. In comparison, the genetic algorithm took an average of 5.15 seconds to find the optimal path, or roughly a 20700\% improvement.

While the absolute execution time of the algorithm (in seconds) is arbitrary and based on algorithm implementation, an improvement in execution time exceeding two orders of magnitude show a very conclusive improvement in performance over the wavefront algorithm. Both of these tests were performed using Matlab on a consumer grade desktop computer. With execution times in the seconds, it is possible that this algorithm could be implemented in real-time on a robotic manipulator whose environment was liable to change infrequently.

The algorithm was found to converge to a solution within an average of 12.3 generations, with a standard deviation of 2.58 generations. This represents a fast convergence of the algorithm to a single solution.

%% Collisions
It was expected to be able to develop an algorithm that would find a valid path which never collided with an obstacle. The results from the experiment show that the algorithm found a collision-free path 565 times in 600 (94\%) of the time. While this was not the anticipated result, it still constitutes a successful solution to the problem. In an industry application, if the algorithm produces a path which collides with an obstacle, it would simply re-initialize the population and start the evolution process again. With a run time of 5.15 seconds, restarting the algorithm would not be prohibitive for most applications.

%% Jerk
Since trajectories varied significantly between environments and start/end locations, a reference best-case scenario jerk could not be identified. The jerk of a path was therefore compared to another instance of the GA where jerk compensation was excluded from the fitness function. This allowed the effect of the jerk reduction to be quantified against a reference path, given a constant environment. It was found that the algorithm reduced the path's maximum jerk by an average of 192\%. Ultimately, the actual jerk of the robotic manipulator will be a function of the robot's speed, however by reducing path jerk, the robot can navigate its trajectory at a higher speed without exceeding allowable jerk.