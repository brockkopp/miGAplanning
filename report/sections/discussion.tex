% CRITERIA (Half a page)
	% - Main findigs
	% - What results mean
	% - Limitations of our method
	% - Comparison of our results (Campare to probabilistic path planning)
	% - Need for further studies
	
% OUTLINE
% - Main findings
	% - Worse distance than wavefront *complete (could use more info?)
	% - Expect success in reducing jerk *complete
		% - Find a metric
		% - Find a gold standard
	% - 100% success of finding a path w/o collisions
	% - Slower than wavefront *complete
	% - num generation *complete
<<<<<<< HEAD
=======

>>>>>>> bcb99cb940c889ffd72195bbb00ac80b39f7c886
%% Length
Evaluating the performance of the genetical algorithm (GA) was complicated by its ability to consider many factors when determining the best path. These factors included path distance, vehicle dynamics and most importantly, obstacle avoidance. In order to determine a profile of the GA's performance, 600 tests were completed over the 15 random point configurations in 3 different environments.

The length of the path generated by the GA approach was on average 0.7\% longer than the path generated by the wavefront method to which it was compared. The wavefront is a deterministic environment which will find a very short path around obstacles, with no consideration for vehicle dynamics. The GA works differently in that it is a stochastic algorithm that is sampling the environment in an effort to find the shortest path. As such, it can get stuck in a local minimum and not ultimately return the path of best fitness. While processes internal to the GA attempt to avoid these situations, it is not a perfect algorithm.

The GA was able to determine a shorter path than the wavefront algorithm 474 of 600 (79\%) independent tests, despite attempting to also minimize jerk (sometimes at the cost of a longer path). While the GA is able to find a shorter path around obstacles 79\% of the time, it is still on average 0.7\% longer than wavefront since occasionally it will have an extrememly long path. These long paths are a result of the GA finding a local minimum and getting stuck there. In an actual implementation, the GA could be run multiple times in an effort to filter out local minima.

%% Speed
The GA was found to be much quicker at finding a solution than the wavefront algorithm. The wavefront algorithm took an average of 17 minutes and 50 seconds to analyze the shortest path between two points on any of the three test spaces used in this study. In comparison, the genetic algorithm took an average of 5.15 seconds to find the optimal path (20767\% improvement). While the actual algorithm execution time is abitrary and based on algorithm implementation, three orders of magnitude faster shows that the GA is conclusively faster than the wavefront algorithm. With execution times in the seconds, it is not unfathomable that this algorithm could be implemented live on a robotic manipulator who's environment was liable to change infrequently.

The algorithm was found to convege to a solution on average within 12.3 generations, with a standard deviation of 2.58 generations. This represents a fast convergence of the algorithm to a single solution.

%% Collisions
If was expected to be able to develop an algorithm that would find a valid path which never collided with an obstacle. The results from the experiment show that the algorithm infact finds a collision-free path 565 times in 600 (94\%) of the time. While this was not the anticipated result, it still constitutes a valid solution to the problem. In an industry application, if the algorithm produces a path which collides with an obstacle, it would simply re-initialize polulation. With a run time of 5.15 seconds, restarting the algorithm would not be prohibitive for most applications.

%% Jerk
Since trajectories varied between environments and start/end location, a reference jerk could not be identified. The jerk of a path was therefore compared to another instance of the GA where jerk was excluded from the fitness function. This allowed the effect of the jerk reduction to be quantified against a reference given a constant environment. It was found that the algorithm reduced path jerk by an average of 192\%. Ultimately, the actual jerk of the robotic manipulator will be a function of the robot's speed, however by reducing path jerk, the robot can navigate its trajectory at a higher speed without exceeding allowable jerk.



