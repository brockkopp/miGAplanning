% CRITERIA	(1 Page)
	% 1.Describe the problem
	 % - Use Examples
	 % - Present the general case
	% 2.State your contributions
		% -Present a list of contributions

Path planning is a critical component of navigation technology, a fundamental area in robotics research. Verifying path safety is critical for both the welfare of the robot as well as its surrounding environment, which can contain people or valuable infrastructure. The path must ultimately reach its destination while satisfying specified criteria, the most common of which is distance (or time). Another important criterion is path dynamics, where the forces and other dynamic characteristics of the robot are considered. Excessive acceleration and jerk cause stress on the robotic manipulator as well as robot tracking characteristics which are often a function of acceleration. These criteria are accounted for by relatively few common path planning algorithms \cite{elshamli04}.

Path planning techniques can be applied to industrial robotic manipulators to ensure efficient operation while maneuvering safely in the environment. Articulated manipulators are bounded to their configuration workspace, which is a two-dimensional representation of their allowable motion range \cite{kavraki96}. Frequently used algorithms for motion planning include probabilistic roadmaps, potential field methods, and neural network approaches \cite{sharir89,khosla88,rimon92,yang00}. Since every robot has a unique and complex task and environment, a robust solution is necessary. Genetic algorithms are especially suited for such complex optimization problems \cite{renner03}.

As a stochastic optimization technique, it is expected that a genetic algorithm will find the global optimum for a problem in a reasonable amount of time and computational cost. Due to the iterative evolutionary nature of genetic algorithms however, computational time may still be greater than other probabilistic algorithms. Overarching these concerns, the robustness of the genetic algorithm, combined with its ability to easily evaluate all path criteria of the optimization problem make it an appealing solution.

The robot's environment and obstacles will be represented in a discretized grid, and polynomial curve fitting techniques will be used to define the optimum path from the start to the end location through the given environment. The curve fitting method helps simplify and expedite the generation of a desired path, compared to commonly used techniques such as rapidly exploring random trees \cite{rodriguez06}. The genetic algorithm will determine the polynomial coefficients which result in a curve with the greatest fitness, as determined by the constraints of the problem (namely path distance and jerk). A polynomial function was selected to describe the path since it inherently results in a continuous trajectory rather than traditional discontinuous waypoint based trajectories. 

While there is no absolute best path planning algorithm for all situations, it is proposed that:
\begin{itemize}
	\item A Genetic Algorithm can effectively plan a path through a robotic manipulator's configuration space.
	\item The algorithm presented can plan a path which not only minizes distace travelled, but also account for robotic manipulator dynamics such as acceleration and jerk.
	\item The algorithm can identify a viable solution which is within XXXX\% of the shortest distance as determined using a deterministic algorithm such as wavefront, but in much less time.
\end{itemize}

% Initial experimentation will be done using lower order splines as suggested paths, but higher-dimensional splines will be studied as well. Extendibility of the suggested approach to 3D environments will be considered. To verify the effectiveness of the algorithm in 2D, the results will be compared to the solution from the wavefront method. The nature of the wavefront algorithm guarantees that it yields the shortest path, but it does not take dynamic constraints into consideration.