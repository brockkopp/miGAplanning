% CRITERIA	(1 Page)
	% 1.Describe the problem
	 % - Use Examples
	 % - Present the general case
	% 2.State your contributions
		% -Present a list of contributions

Path planning is a critical component of navigation technology, a fundamental area in robotic research. The purpose of path planning is to define a trajectory between specified start and end points. The path must achieve the problem objectives while satisfying specified criteria, the most common of which are distance and time. Other important criteria include smoothness and path safety, and these are considered by relatively few common path planning algorithms \cite{elshamli04}. Optimizing smoothness of a path ensures that the physical and dynamic limitations of the robotic system (e.g. maximum motor torques) are taken into consideration. Verifying path safety is critical for both the safety of the robot as well as its surrounding environment, which can contain people or valuable infrastructure.

Path planning techniques can be applied to industrial robotic manipulators to ensure efficient operation while maneuvering safely in the environment. Articulated manipulators are bounded to their configuration workspace, which is a two-dimensional representation of their allowable motion range \cite{kavraki96}. Frequently used algorithms for motion planning include probabilistic roadmaps, potential field methods, and neural network approaches \cite{sharir89,khosla88,rimon92,yang00}. Since every robot has a unique and complex task and environment, a robust solution is necessary. Genetic algorithms are especially suited for such complex optimization problems \cite{renner03}.

As a stochastic optimization technique, a genetic algorithm will find the global optimum for a problem in a reasonable amount of time and computational cost. Due to the evolutionary nature of genetic algorithms, computational time may be greater than other applicable algorithms. However, the robustness of a genetic algorithm, combined with its elegant mechanisms of incorporating all factors of the optimization problem, make it an appealing solution.

The robot's environment and obstacles will be represented in a discretized grid, and curve fitting techniques will be used to define the optimum path from the start to the end location through the specified environment. The curve fitting method helps simplify and expedite the generation of a desired path, compared to commonly used techniques such as rapidly exploring random trees \cite{rodriguez06}. The genetic approach to curve fitting will select the functional coefficients which result in a curve with the greatest fitness. Fitness will be determined by the constraints of the problem (namely distance and safety).

Initial experimentation will be done using lower order splines as suggested paths, but higher-dimensional splines will be studied as well. Extendibility of the suggested approach to 3D environments will be considered. To verify the effectiveness of the algorithm in 2D, the results will be compared to the solution from the wavefront method. The nature of the wavefront algorithm guarantees that it yields the shortest path, but it does not take dynamic constraints into consideration.