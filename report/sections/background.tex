% CRITERIA	(1 page)
	% ?Materials & Method
	% ?Datasets
	% ?Materials (sources)
	% ?Methods (reference and brief description)
	% ?Proposed approach: algorithm
	% ?Equations and formulas

Path planning is an active area of research due to the complexity of the problem and the need for a robust solution applicable to various environments. Many algorithms exist, but none of the proposed algorithms are capable of encompassing all the problem constraints for various environments \cite{sariff06}.

Existing path planning solutions include reactive motion planning algorithms, graph and probabilistic based motion planning, and optimization based planning \cite{waslanderI}. Each of these has advantages and disadvantages with regards to planning a route through a robot's configuration space. Reactive planning algorithms such as potential field and wavefront methods are simple in concept however these are local approach methods and do not consider dynamic constraints. Some are inherently susceptible to local minima, making them unacceptable for obstacle avoidance \cite{koren91}. The wavefront method is designed to find the shortest path, and as such will be used as the benchmark against which results from the suggested algorithm will be measured. It is expected that while the suggested algorithm will generate a longer path, the dynamic constraints will be satisfied unlike in the wavefront solution.

Graph based planning offers a global approach. Graphs may be generated both deterministically (visibility graphs, cell decomposition, Voronoi diagrams), or randomly (probabilistic roadmaps). [10] Established algorithms (such as Dijkstra's shortest path search) are used to extract the shortest path from the graph \cite{dijkstra59}. However dynamics of the robot are ignored in the generation of the graph, and this is detrimental to the criterion of path smoothness. Challenges also arise when there are restrictive obstacles such as narrow passages, however research in areas such as probabilistic roadmaps (PRMs) offers promising solutions \cite{hsu03}. Optimization based planning offers more confidence in finding a global optimum, however it is hindered by a lack of robustness: a poorly formulated problem may not converge \cite{waslanderIII}. It is also difficult to define obstacles.

In an effort to provide a simple method of avoiding obstacles and defining a path, the manipulator's configuration space will be discretized using a grid-based approach, and curve fitting techniques will be employed to find the ideal path. Defining the environment with a grid-graph may not be the most effective (cell decomposition offers greater efficiency \cite{lingelbach04}), however it provides a simple way to defined obstacles as well as flexibility in representation. The grid will represent the configuration space for an articulated industrial robot. The grid cell size will define the computational resources required. A tradeoff between resolution in the grid and computational efficiency must be made.

To define the fitted curve, research has been done using B�zier curves \cite{choi10}. B�zier curves provide smooth paths that guarantee obstacle avoidance. While very effective for lower-dimensional problems, quartic or quintic functions begin to get very computationally expensive. That is why this paper aims to use curve fitting in combination with genetic algorithms to optimize the curve. The curve will be optimized by minimizing distance, maximizing smoothness (minimizing jerk), and ensuring a safe passage around all obstacles. Genetic algorithms are ideal for such complex optimization problems and offer good robustness as well \cite{zou12}.

The father of genetic algorithms, John Holland, modeled this machine intelligence technique after the evolution research by Darwin and the genetic ``survival of the fittest'' discovered in the natural, biological sphere. There are two natural learning epitomes available: the brain and evolution. Holland was inspired by the fact that the processes of natural evolution and natural genetics have become elucidated over decades of study in biology and molecular biology \cite{goldberg88}. The subtleties of the fundamental mechanisms of the brain, in contrast, are still shrouded in mystery. As such it seems clear to use the better understood model of genetic evolution as a platform for an optimization technique.

Research has already been done regarding the application of the parallel heuristic search method of genetic algorithms to curve fitting \cite{gulsen95,karr91,ismail08}. The goal is to find the coefficients of a polynomial function that will define a path that with optimized path criteria. Thus each chromosome will contain a set of function parameters, labelled $\{P_4, \ldots, P_0\}$ in Equation \ref{eq:polyTraj}.

\begin{align} \label{eq:polyTraj}
	\phi = P_4*\theta^4 + P_3*\theta^3 + P_2*\theta^2 + P_1*\theta + P_0
\end{align}

Testing will be performed by creating random environments containing arbitrarily place objects. The algorithm will then plot a trajectory between a randomly defined start and end point. Many iterations of these tests will be performed in a variety of environments in order to accurately determine the effectiveness of this algorithm. The average path distance for each environment will be compared to the minimum distance as calculated by a Wavefront algorithm (CITE).

The wavefront algorithm is a deterministic algorithm which is capable of determining the shortest path between two points through a descritized environment. This will provide a ``gold standard'' which can be used to evaluate the performance of the genetical algorithm. Since the genetic algorithm is a stochastic search, it will never result in an identical path twice. As such all results will be presented as averages and their associated distributions.

The result will be a global path planning algorithm functional in a static environment. Dynamic environments are possible with the proposed algorithm, however computational efficiency must be optimized in order to minimize the time required to find the solution. This new applications offers a simple way to determine a path while taking into consideration all the path criteria.