The results of path planning using genetic algorithms presented in this report show that genetic algorithms can effectively plan a path through a series of obstacles, and present an opportunity to improve on existing path planning techniques. By selecting the appropriate fitness criteria, it is possible to bias a genetic search algorithm to find a path which satisfies multiple geometric conditions. In this implementation, the path was optimized to have a short path length, no collisions and minimal jerk.

The algorithm is on average able to produce a path that is roughly the same length (100.7\%) as the shortest path as determined by the wavefront search method, while improving comptutation times by a factor of 100. 79\% of the time the algorithm finds a path that is shorter than that produced by the wavefront method, with much longer paths occuring occasionally (in the range of 0.1\% to 0.5\% of the time). Collision avoidance is of critical importance in the application of path planning to industrial manipulators. This implementation of genetic algorithms could produce a collision free path 94\% of the time. Finally, it was shown that the algorithm's jerk compensation was able to reduce path jerk by 11\%, when compared to the same algorithm without jerk compensation.

Through a transformation from Cartesian space to the manipulator's join configuration space, a method has been suggested to apply these path planning techniques to obstacle avoidance for industrial manipulators.

A main limitation of this algorithm is that since it makes use of ordered pair polynomial functions, there is a limited geography through which it is possible to evolve (a single 'y' value per 'x' value). Invalid solutions would therefore be found by the algorithm in a situation where it necessary for the path to reverse upon itself to avoid an obstacle. This is a fundamental limitation of a polynomial. In order to overcome this limitation, a function family would need to be selected that could double back across the same x-axis location.

Future study should be considered to determine the optimal function order for varying degrees of environment complexity. Increasing function order will allow for a more complex path, but at the cost of increased complexity and computational requirements. Further expansion of this technique into three degree-of-freedom and greater manipulators will greatly improve upon the industrial applicability of this technique. Validation of realizable improvements to path dynamics is also necessary to validate the effectiveness of jerk and other dynamic compensation techniques.