The results of the implementation of path planning using genetic algorithms presented in this report show that genetic algorithms can effectively plan a path through a series of obstacles, and present a opportunities to improve on existing path planning techniques. By selecting the appropriate fitness criteria, it is possible to bias your genetic search algorithm to find a path which satisfies multiple geometric conditions. In this implementation, the path was optimized for short length, no collisions and low jerk. 

The algorithm is on average able to produce a path that is 100.7\% the length of the shortest path as determined by the wavefront search method. 79\% of the time, the algorithm can find a path that is shorter than that produced by the wavefront method. Collision avoidance is of critical importance in the application of path lanning to industrial manipulators. This implemenation of genetic algorithms could produce a collision free path 85\% of the time. Finally, it was shown that the jerk of a path could be reduced by 11\% when the fitness function included jerk criteria.

Through a manipulation from cartesian space to joint space to a configuration space, a method has been suggested to apply these path planning techniques to obstacle avoidance for industrial manipulators.

A main limitation of this algorithm is that the implementation using polynomials has a limited geography through which it is possible to evolve. Invalid solutions would be found by the algorithm whenever it was necessary for the path to double back, so that there are two y-axis values for one x-axis value. This is a fundamental limitation of a polynomial. In order to overcome this limitation, a function family would need to be selected that could double back across the same x-axis location.